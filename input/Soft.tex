
A Software Requirements Analysis for a software system is a complete 
description of the behavior of a system to be developed. It include functional Requirements
and Software Requirements. In addition to these, the SRS contains 
non-functional requirements. Non-functional requirements are 
requirements which impose constraints on the design or implementation.
\begin{itemize}
\item{\bf Purpose}: WebOctave and Numerical Methods Analysis Tool are built with purpose:
\begin{enumerate}
\item  To Perform most of difficult Calculation work.
\item To work remotely without installaton of Ocatve.
\item To plot graphs and download them.
\end{enumerate}
\newpage
\item{\bf Users of the System}

 Researcher or Student-: They have knowledge of working of procedures and what input is being provided.  
WebOctave can be installed and used by anyone who wants to perform any numerical computation using Octave.


\subsection{Functional Requirements}
\item {\bf Specific Requirements}: This phase covers the whole requirements 
for the system. After understanding the system we need the input data 
to the system then we watch the output and determine whether the output 
from the system is according to our requirements or not. So what we have 
to input and then what we’ll get as output is given in this phase. This 
phase also describe the software and non-function requirements of the 
system.
\item {\bf Input Requirements of the System}
\begin{enumerate} 
\item Guess points
\item Precision
\item Step-size in case of iterative methods.
\item Required point at which value is to be found
\end{enumerate}
\vskip 0.5cm
\item {\bf Output Requirements of the System}
\begin{enumerate} 
\item Final output after iterations.
\item Graphs wherever possible in form of images. 
\end{enumerate}
\vskip 0.5cm
\item {\bf Software Requirements}
\begin{enumerate} 
\item Programming language: Ocatve 4.0
\item software: \LaTeX{}
\item Web Languages: php
\item Database: Mysql 
\item Documentation: Doxygen 1.8.3
\item Text Editor: Vim
\item Operating System: Ubuntu 14.04 or up
\item Revision System: Git
\end{enumerate}
\end{itemize}


