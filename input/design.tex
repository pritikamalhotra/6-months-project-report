
\section{Facilities required for proposed work}
\subsection{Hardware Requirements}
\begin{itemize}
\item Operating System: Linux/Windows
\item Processor Speed: 512KHz or more
\item RAM: Minimum 256MB
\end{itemize}
\subsection{Software Requirements}
\begin{itemize}
\item Software: GNU Ocatve, Gnuplot(for plots if not using default toolkit),git(version control).
Weboctave requires php, mysql and apache for it's installation.
\item Programming Language: Octave
\end{itemize}

\section{Methodology}
\begin{itemize}
\item Studying various methods available to solve different problems of numerical analysis.
\item Deciding various input and output parameters of methods.
\item Making the approach modular 
\item Graphical representation of solutions wherever possible
\item Generating documentation
\end{itemize}
%\newpage
\section{Project Work} 
\textbf{Studied Previous System:}\\
Before starting the project, \\\\
\textbf{Learn octave:}\\
Before starting with project, we have to go through the basics of Octave, such that there
should not be any problem proceeding with project.\\\\
\textbf{Get Familiar with Different methods and their algorithms:}\\
We have gone through algorithms of these algorithms. Then implementation becomes easy\\\\
\textbf{Functions:}\\
The user has been provided some test functions which he can use to test various.\\\\
\textbf{Plots:}\\
Octave provides fltk as the default toolkit. But we can use gnuplot for more accurate plotting by setting them as default toolkit.\\\\
\textbf{Input:}\\
Input values are taken from user or default values defined in the file are used.\\\\
\textbf{Output:}\\
The iterations are performed and it returns the output with the expected precision.\\\\


