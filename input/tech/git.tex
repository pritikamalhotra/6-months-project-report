
\section{Introduction to Github}
%\begin{figure}[!ht]
%\centering
%\includegraphics[width=0.3\textwidth]{images/github}                   
%\caption{Github Logo}
%\hspace{-1.5em}
%\end{figure}
%\leavevmode\\
GitHub is a Git repository web-based hosting service which offers all of the functionality of Git as well as adding many of its own features. Unlike Git which is strictly a command-line tool, Github provides a web-based graphical interface and desktop as well as mobile integration. It also provides access control and several collaboration features such as wikis, task management, and bug tracking and feature requests for every project.\\

GitHub offers both paid plans for private repto handle everything from small to very large projects with speed and efficiency. ositories, and free accounts, which are usually used to host open source software projects. As of 2014, Github reports having over 3.4 million users, making it the largest code host in the world.\\

GitHub has become such a staple amongst the open-source development community that many developers have begun considering it a replacement for a conventional resume and some employers require applications to provide a link to and have an active contributing GitHub account in order to qualify for a job.\\

The Git feature that really makes it stand apart from nearly every
other Source Code Management (SCM) out there is its branching model.\\
\\
Git allows and encourages you to have multiple local branches that can
be entirely independent of each other. The creation, merging, and
deletion of those lines of development takes seconds.\\ \\
This means that you can do things like:
\begin{itemize}
\item Frictionless Context Switching.\\ Create a branch to try out an
idea, commit a few times, switch back to where you branched from,
apply a patch, switch back to where you are experimenting, and merge
it in.
\item Role-Based Code lines. \\ Have a branch that always contains only
what goes to production, another that you merge work into for testing,
and several smaller ones for day to day work.
\item Feature Based Work flow. \\ Create new branches for each new
feature you're working on so you can seamlessly switch back and forth
between them, then delete each branch when that feature gets merged
into your main line.
\item Disposable Experimentation.\\  Create a branch to experiment in,
realize it's not going to work, and just delete it - abandoning the
work—with nobody else ever seeing it (even if you've pushed other
branches in the meantime).
\end{itemize}
Notably, when you push to a remote repository, you do not have to push
all of your branches. You can choose to share just one of your
branches, a few of them, or all of them. This tends to free people to
try new ideas without worrying about having to plan how and when they
are going to merge it in or share it with others.\\ \\
There are ways to accomplish some of this with other systems, but the
work involved is much more difficult and error-prone. Git makes this
process incredibly easy and it changes the way most developers work
when they learn it.

\subsection{What is Git?}
%\begin{figure}[!ht]
%\centering
%\includegraphics[width=0.3\textwidth]{images/git}                   
%\caption{Git Logo}
%\hspace{-1.5em}
%\end{figure}
Git is a distributed revision control and source code management (SCM) system with an emphasis on speed, data integrity, and support for distributed, non-linear workflows. Git was initially designed and developed by Linus Torvalds for Linux kernel development in 2005, and has since become the most widely adopted version control system for software development.\\

As with most other distributed revision control systems, and unlike most client–server systems, every Git working directory is a full-fledged repository with complete history and full version-tracking capabilities, independent of network access or a central server. Like the Linux kernel, Git is free and open source software distributed under the terms of the GNU General Public License version 2 to handle everything from small to very large projects with speed and efficiency.\\

Git is easy to learn and has a tiny footprint with lightning fast performance. It outclasses SCM tools like Subversion, CVS, Perforce, and ClearCase with features like cheap local branching, convenient staging areas, and multiple workflows.\\

\subsection{Installation of Git}

Installation of git is a very easy process.
The current git version is: 2.0.4.
Type the commands in the terminal:\\\\
\emph{
\$ sudo apt-get update\\\\
\$ sudo apt-get install git\\\\}
This will install the git on your pc or laptop.

\subsection{Various Git Commands}

Git is the open source distributed version control system that facilitates GitHub activities on your laptop or desktop. The commonly used Git command line instructions are:-\\

\subsubsection{Create Repositories}
Start a new repository or obtain from an exiting URL

\begin{description}

\item [\$ git init [ project-name]]\\
Creates a new local repository with the specified name
\item [\$ git clone [url]]\\
Downloads a project and its entire version history\\

\end{description}


\subsubsection{Make Changes}
Review edits and craft a commit transaction

\begin{description}

\item [\$ git status] \leavevmode \\
Lists all new or modified files to be committed

\item [\$ git diff] \leavevmode \\
Shows file differences not yet staged

\item [\$ git add [file]]\\
Snapshots the file in preparation for versioning

\item [\$ git reset [file]]\\
Unstages the file, but preserve its contents

\item [\$ git commit -m "[descriptive message]"]\\
Records file snapshots permanently in version history\\

\end{description}


\subsubsection{Group Changes}
Name a series of commits and combine completed efforts

\begin{description}

\item [\$ git branch] \leavevmode \\
Lists all local branches in the current repository

\item [\$ git branch [branch-name]]\\
Creates a new branch

\item [\$ git checkout [branch-name]]\\
Switches to the specified branch and updates the working directory

\item [\$ git merge [branch]]\\
Combines the specified branch’s history into the current branch

\item [\$ git branch -d [branch-name]]\\
Deletes the specified branch\\

\end{description}


\subsubsection{Save Fragments}
Shelve and restore incomplete changes

\begin{description}

\item [\$ git stash] \leavevmode \\
Temporarily stores all modified tracked files

\item [\$ git stash pop] \leavevmode \\
Restores the most recently stashed files

\item [\$ git stash list] \leavevmode \\
Lists all stashed changesets

\item [\$ git stash drop] \leavevmode \\
Discards the most recently stashed changeset\\

\end{description}


\subsubsection{Synchronize Changes}
Register a repository bookmark and exchange version history

\begin{description}

\item [\$ git fetch [bookmark]]\\
Downloads all history from the repository bookmark

\item [\$ git merge [bookmark]/[branch]]\\
Combines bookmark’s branch into current local branch

\item [\$ git push [alias][branch]]\\
Uploads all local branch commits to GitHub

\item [\$ git pull] \leavevmode \\
Downloads bookmark history and incorporates changes

\end{description}
