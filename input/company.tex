\section{Introduction to organisation}
\begin{figure}[ht]
\centering
\includegraphics[scale=0.5]{images/gndec.jpg}
\caption{Guru Nanak Dev Engineering College}
\end{figure}
\hspace{-1.7em} I had my Six Months Industrial Training at TCC-Testing And Consultancy Cell, GNDEC Ludhiana. Guru Nanak Dev Engineering College was established by the Nankana
Sahib Education Trust Ludhiana. The Nankana Sahib Education Trust i.e NSET
was founded in memory of the most sacred temple of Sri Nankana Sahib, birth place
of Sri Guru Nanak Dev Ji. With the mission of Removal of Economic Backwardness
through Technology Shiromani Gurudwara Parbandhak Committee i.e SGPC started a
Poly technical was started in 1953 and Guru Nanak Dev Engineering College was established in 1956.\\

NSET resolved to uplift Rural areas by admitting 70\% 
of students from these rural
areas ever year. This commitment was made to nation on 8th April, 1956, the day
foundation stone of the college building was laid by Dr. Rajendra Prasad Ji, the First
President of India. The College is now ISO 9001:2000 certified.\\

Guru Nanak Dev Engineering College campus is spread over 88 acres of prime land
about 5 Km s from Bus Stand and 8 Kms from Ludhiana Railway Station on Ludhiana-Malerkotla Road. The college campus is well planned with beautifully laid out tree plantation, pathways, flowerbeds besides the well maintained sprawling lawns all around. It
has beautiful building for College,Hostels,Swimming Pool,Sports and Gymnasium Hall
Complex, Gurudwara Sahib, Bank, Dispensary, Post Office etc. There are two hostels
for boys and one for girls with total accommodation of about 550 students. The main
goal of this institute is:\\
\begin{itemize}
\item To build and promote teams of experts in the upcoming specialisations.
\item To promote quality research and undertake research projects keeping in view their
relevance to needs and requirements of technology in local industry.
\item To achieve total financial independence.
\item To start online transfer of knowledge in appropriate technology by means of establishing multipurpose resource centres.
\end{itemize}
\section{Testing and Consutancy Cell}

My Six Months Institutional Training was done by me at TCC i.e Testing And
Consultancy Cell,
GNDEC Ludhiana under the guidance of Dr. H.S.Rai Dean Testing and Consultancy Cell.
Testing and Consultancy Cell was established in the year 1979 with a basic aim to produce
quality service for technical problems at reasonable and affordable rates as a service to society
in general and Engineering fraternity in particular.\\
\begin{figure}[ht]
\centering
\includegraphics[scale=0.7]{images/aw.jpg}
\caption{Testing and Consultancy Cell}
\end{figure}
\hspace{-1.7em} 

Consultancy Services are being rendered by various Departments of the College to the
industry, Sate Government Departments and Entrepreneurs and are extended in the form of
expert advice in design, testing of materials \& equipment, technical surveys, technical audit,
calibration of instruments, preparation of technical feasibility reports etc.
This consultancy cell of the college has given a new dimension to the development
programmers of the College. Consultancy projects of over Rs. one crore are completed by the
Consultancy cell during financial year 2009-10. \\

Ours is a pioneer institute providing Consultancy Services in the States of Punjab, Haryana,
Himachal, J\&K and Rajasthan. Various Major Clients of the Consultancy Cell are as under:\\
\begin{itemize}
\item Northern Railway, Govt. of India
\item Indian Oil Corporation Ltd.
\item Larson \& Turbo.
\item Multi National Companies like AFCON \& PAULINGS.
\item Punjab Water Supply \& Sewage Board
\item Power Grid Corporation of India.
\item National Building Construction Co.
\item Punjab State Electricity Board.
\item Punjab Mandi Board.
\item Punjab Police Housing Corporation.
\item National Fertilizers Ltd.
\item GLADA, Ludhiana
\end{itemize}